%%%%%%%%%%%%%%%%%%%% author.tex %%%%%%%%%%%%%%%%%%%%%%%%%%%%%%%%%%%
%
% sample root file for your "contribution" to a proceedings volume
%
% Use this file as a template for your own input.
%
%%%%%%%%%%%%%%%% Springer %%%%%%%%%%%%%%%%%%%%%%%%%%%%%%%%%%


\documentclass{svproc}
%
% RECOMMENDED %%%%%%%%%%%%%%%%%%%%%%%%%%%%%%%%%%%%%%%%%%%%%%%%%%%
%

% to typeset URLs, URIs, and DOIs
\usepackage{url}
\def\UrlFont{\rmfamily}

\begin{document}
\mainmatter              % start of a contribution
%
\title{Efficient Algebraic Multigrid Preconditioners \\on Clusters of GPUs}
%
%\titlerunning{Hamiltonian Mechanics}  % abbreviated title (for running head)
%                                     also used for the TOC unless
%                                     \toctitle is used
%
\author{Ambra Abdullahi Hassan\inst{1}%
\and Valeria Cardellini\inst{1} \and Pasqua D'Ambra\inst{2} \and \\ 
Daniela di Serafino\inst{3} \and  Salvatore Filippone\inst{4}}
%
%\authorrunning{} % abbreviated author list (for running head)
%
%%%% list of authors for the TOC (use if author list has to be modified)
\tocauthor{Ivar Ekeland, Roger Temam, Jeffrey Dean, David Grove,
Craig Chambers, Kim B. Bruce, and Elisa Bertino}
%
\institute{Universit\`a degli Studi di Roma ``Tor Vergata'', Roma, Italy\\
 \email{\{ambra.abdullahi,cardellini\}@uniroma2.it}
\and
Istituto per le Applicazioni del Calcolo ``Mauro Picone'', CNR, Napoli, Italy\\
 \email{pasqua.dambra@cnr.it}
\and
Universit\`a degli Studi della Campania ``Luigi Vanvitelli'', Caserta, Italy
\email{daniela.diserafino@unicampania.it}
\and
 Cranfield University, Cranfield, UK\\
 \email{salvatore.filippone@cranfield.ac.uk}
}
\maketitle              % typeset the title of the contribution

\begin{abstract}
Many scientific applications require the solution of large and sparse linear systems of equations using iterative methods, this operation accounting for a large percentage of the computing time. In these cases, the choice of the preconditioner is crucial for the convergence of the iterative method.
Given the broad range of applications, great effort has been put in the development of efficient  preconditioners ensuring algorithmic scalability: in this sense, multigrid methods have been proved to be particularly promising.
Additionally, the advent of GPUs, now found in many of the fastest supercomputers, poses the problem of implementing efficiently these algorithms on highly parallel architectures; this is made more difficult by the fact that the solution of sparse triangular systems, a common kernel in many types of preconditioners, is extremely inefficient on GPUs. 

In this paper, we use the PSBLAS and MLD2P4 libraries to explore various issues that affect the efficiency of multilevel preconditioners on GPUs, both in terms of execution speed and exploitation of computational cores, as well as the algorithmic efficiency in guaranteeing convergence to solution in a number of iterations independent of the parallelism degree. 
We investigate these issues 
%are investigated 
in the context of linear systems arising from groundwater modeling application of the filtration of 3D incompressible single-phase flows through %anisotropic 
porous media.

\end{abstract}
%
\section{Introduction}
%
%
\section{Multilevel preconditioner}
%
Multilevel preconditioner e nuclei computazionali dell’algoritmo di applicazione.
 \section{Sparse linear algebra on GPUs}
 Sparse linear algebra on GPU e limitazioni del nucleo risoluzione di sistemi triangolari, uso delle inverse approssimate e focus sulla INVK di PSBLAS.


% \section{Sparse linear algebra on GPUs}
% Sparse linear algebra on GPU e limitazioni del nucleo risoluzione di sistemi triangolari, uso delle inverse approssimate e focus sulla INVK di PSBLAS.
\section{Results}
Distribuzione dati fatta tenendo conto della geometria (3D) del problema e giustificandola rispetto al vantaggio che fornisce in termini di rapporto calcolo/comunicazione
\paragraph{Aacknowledgement}


%
% ---- Bibliography ----
%
\begin{thebibliography}{6}
%

\bibitem {}


\end{thebibliography}
\end{document}
