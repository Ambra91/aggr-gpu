\section{Results}
(Distribuzione dati fatta tenendo conto della geometria (3D) del problema e giustificandola rispetto al vantaggio che fornisce in termini di rapporto calcolo/comunicazione.)

We illustrate the behaviour of the AMG preconditioner and the INVK solver on GPUs  by showing the results obtained on a linear system arisen from a groundwater modelling application developed at the Juelich Supercomputing Centre (JSC)  dealing with numerical simulation of the filtration of 3D incompressible single-phase flows through anisotropic porous media. It is an elliptic equation with no flow boundary conditions. The linear systems arises from the discretization of the equation performed by a cell-centered finite colume scheme (two-point flux approximation) on a Cartesian grid, with nonzero entries distributed over seven diagonals. In these tests, we will consider a homogeneous permeability tensor. In the following we will consider matrice with dimensions ranging from 1 million to 256 millions of equations.  
This application comes from the framework of the Horizon 2020 EoCoE Project.

We ran experiments on the JURECA supercomputer at the Juelich Supercomputing Centre (JSC).  Each GPU compute node consists of two NVIDIA Tesla K80 GPUs with a dual-GPU design, for a total of four available GPU devices per compute nodes. 
We used GCC/5.4.0 and MVAPICH2/2.3 with CUDA 8.0.61.
