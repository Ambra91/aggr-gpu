\section{Sparse Linear Algebra and Preconditioners on GPUs}

General Purpose Graphics Processing Units (GPGPUs)~\cite{Luebke06}  
are today an established and attractive choice in the world of
scientific computing, found in many among the fastest supercomputers
on the Top~500 list, and offered in standard Cloud infrastructure
services, e.g. in  Amazon EC2. It is therefore not surprising that
they are at the focus of many research efforts revolving around the
efficient implementation of scientific software, including solvers for
sparse linear system of equations. 

Many applications use iterative methods to solve sparse linear
systems;  the most popular ones are those based on  Krylov subspace 
projection methods~\cite{MR1990645}. 

From a  software point of view, all Krylov methods employ the matrix $A$ 
to perform matrix-vector products  $y\gets Ax$, whose efficient
implementation provides significant challenges on all modern computer
architectures. 
The SpMV kernel is well-known to be a memory-bounded application;
and  its bandwidth usage  is strongly dependent on both the input
matrix and on the underlying computing platform(s). For a detailed
overview of the implementation issues and a survey of available
techniques for this kernel on GPUs, see~\cite{Filippone:2017:SMM:3034774.3017994}.   

For most real-world problems,it is necessary to combine the Krylov
solvers with  \emph{preconditioners}. A preconditioner is a
transformation applied to the coefficient  matrix such that the
resulting linear system has better convergence properties. Examples of
popular preconditioners include Jacobi and Gauss-Seidel iterations,
Block Jacobi and Additive Schwarz coupled with incomplete
factorizations,  and Algebraic Multigrid. 

To implement a linear system solver on a parallel computing
architecture, we typically use a \emph{domain decomposition} approach,
in which subsets of the computational domain are assigned to different
computing nodes. Domain decomposition requires to achieve a number of
trade-offs among multiple factors:
\begin{itemize}
\item Many preconditioners, e.g. Gauss-Seidel or incomplete
  factorizations, employ kernels for the solution to sparse triangular
  systems; in using sparse triangular solvers, we normally  want the
  sparsity pattern to be of the same order as that of the original
  coefficient matrix;
\item However, the amount of parallelism available in a sparse
  triangular solution is dependent on the number of nonzeros in the
  sparse factors, and is usually very small;
\item To overcome this issue, most preconditioners use a Block Jacobi
  or hybrid Gauss-Seidel strategy, in which the sparse triangular
  solve is used only within the local subdomain, while employing a
  global Jacobi-like correction;
\item Unfortunately, the convergence properties of these local schemes
  deteriorate with increased parallelism.
\end{itemize}
The last phenomenon is often described as a loss of \emph{algorithmic
  scalability}, i.e., the ability to obtain a convergence history that
is independent of the degree of parallelism. In summary, simple
preconditioning schemes, such as Point Jacobi, have good algorithmic scalability,
because their convergence properties do not change much with an
increasing number of processes, but the more sophisticated
Gauss-Seidel and incomplete factorizations are a lot more efficient in
serial computations. 

Multilevel corrections, or Algebraic Multigrid  preconditioners, offer
a way to recover algorithmic scalability: 
it is possible to apply efficient local approximate solvers, while a
good convergence history is recovered by relying on the multilevel
corrections. 

Finding the preconditioning combination that gives the best overall
performance is thus a very complex enterprise. In this context, the
ability to combine multiple options in a flexible framework such as
the one we described in~\cite{mld-toms} is an essential ingredient. 

Additional considerations are necessary when implementing
preconditioning operations on GPUs. First of all, implementing the
solution of sparse triangular systems on GPUs is extremely difficult.
As an example, the conjugate gradient preconditioned with incomplete
LU available in CUDA CuSPARSE since version 4.0~\cite{Naumov11}
achieves at best a speedup factor of about 2 over a standard CPU
implementation. The main reason is that the GPU requires a massive
amount of data parallelism to be exploited, and in the sparse
triangular factors the amount of available parallelism is limited by
the sparsity. 
The two main  options that  are available  involve the use of
matrix-vector products as their main kernel:
\begin{itemize}
\item Point-Jacobi iterations; 
\item (Local) preconditioning through sparse approximation of the
  inverses.
\end{itemize}
For a discussion of sparse approximate inverses on GPUs 
see~\cite{BERTACCINI2016693}. 


The implementation of the multilevel preconditioners for which we will
present results in this paper is based on MLD2P4~\cite{mld-toms} and
PSBLAS~\cite{psblas3}. Thanks to the modular architecture of both
packages, it is possible to define \emph{plugins} for 
\begin{itemize}
\item Operators on the GPU, including sparse matrix-vector product
  supporting multiple storage formats and vector-vector operations;
\item Approximate inverses as local solvers.
\end{itemize}
By combining these ingredients in a multilevel framework we can obtain
interesting scalability results. 

Three additional practical obstacles must be mentioned here. 
First of all, the preconditioner is invoked at least once per
iteration of the Krylov method.  Because the allocation of memory
on the GPU is an expennse synchronization point, this means that the
memory space for internal work areas must be preallocated for all
invocations  of the preconditioner; allocation of auxiliary data areas at
each invocation of the prevonditioner is perfectly doable on the CPU
side, but leads to a slowdown by a factor of 2 on the GPU. 

The second issue has to do with the implementation of the
matrix-vector product on the GPU: as mentioned
in~\cite{Filippone:2017:SMM:3034774.3017994}, the most effective
storage formats are often derived from ELLPACK  and assign one thread
per row of the matrix; this means that to exploit a GPU it is
necessary to have matrices which are sufficiently large to keep all
computing units busy, and this is very difficult at the coarse level
of the preconditioner hierarchy because the size of the aggregates is
small. Indeed, it may be convenient to limit the number of levels to
be less than the default that would be used in a normal CPU
implementation. 
{\bf DA VERIFICARE CON I DATI DI PERFORMANCE}. 

The third issue is related to the efficiency of communication between
the various computing nodes; to proceed with the computation of the
matrix-vector product in parallel between the different nodes, it is
necessary to exchange the data of the \emph{halo}. For good data
partitions, the amount of data to be exchanged displays a
surface-to-volume effect, and therefore large products can be computed
effectively; however when moving between different levels of the
preconditioning hierarchy we are precisely going towards smaller
matrices, hence the surface-to-volume ratio and the efficiency of the
parallelization decreases. The speed ratio of the GPU with respect to
the CPU only makes things worse; again, it may be necessary to limit
the number of levels to less than what would be efficient when using
CPUs. 